\documentclass[11pt,preprint, authoryear]{article}

\pagestyle{plain}

\usepackage{lmodern}

% spacing passed through from .Rmd doc
\usepackage{setspace}
\setstretch{1.5}

% Wrap around which gives all figures included the [H] command, or places it "here". This can be tedious to code in Rmarkdown.
\usepackage{float}
\let\origfigure\figure
\let\endorigfigure\endfigure
\renewenvironment{figure}[1][2] {
    \expandafter\origfigure\expandafter[H]
} {
    \endorigfigure
}

\let\origtable\table
\let\endorigtable\endtable
\renewenvironment{table}[1][2] {
    \expandafter\origtable\expandafter[H]
} {
    \endorigtable
}

\usepackage{ifxetex,ifluatex}
\usepackage{fixltx2e} % provides \textsubscript
\ifnum 0\ifxetex 1\fi\ifluatex 1\fi=0 % if pdftex
  \usepackage[T1]{fontenc}
  \usepackage[utf8]{inputenc}
\else % if luatex or xelatex
  \ifxetex
    \usepackage{mathspec}
    \usepackage{xltxtra,xunicode}
  \else
    \usepackage{fontspec}
  \fi
  \defaultfontfeatures{Mapping=tex-text,Scale=MatchLowercase}
  \newcommand{\euro}{€}
\fi

\usepackage{amssymb, amsmath, amsthm, amsfonts}

\DeclareMathSizes{24}{26}{22}{22}

\usepackage[round]{natbib}
\bibliographystyle{natbib}
\def\bibsection{\section*{References}} %%% Make "References" appear before bibliography

% package for nice tables
\usepackage{longtable}

% package for ruling lines in tables
\usepackage{booktabs}

% set margins
\usepackage[left=3.5cm, right=2cm, top=30mm ,bottom=2cm, includefoot]{geometry}
\usepackage{fancyhdr}
\usepackage[bottom, hang, flushmargin]{footmisc}
\usepackage{graphicx}
\numberwithin{equation}{section}
%\numberwithin{figure}{section} % commented out because it messes up figure numbering
%\numberwithin{table}{section} % commented out because it messes up table numbering
\setlength{\parindent}{0cm}
\setlength{\parskip}{1.3ex plus 0.5ex minus 0.3ex}
\usepackage{textcomp}
\renewcommand{\headrulewidth}{0pt}

\usepackage{array}
\newcolumntype{x}[1]{>{\centering\arraybackslash\hspace{0pt}}p{#1}}

\usepackage{hyperref}
\hypersetup{breaklinks=true,
            bookmarks=true,
            colorlinks=true,
            citecolor=blue,
            urlcolor=blue,
            linkcolor=blue,
            pdfborder={0 0 0}}
						
\urlstyle{same}  % don't use monospace font for urls
\setlength{\parindent}{0pt}
\setlength{\parskip}{6pt plus 2pt minus 1pt}
\setlength{\emergencystretch}{3em}  % prevent overfull lines
\setcounter{secnumdepth}{5}

% Use protect on footnotes to avoid problems with footnotes in titles
\let\rmarkdownfootnote\footnote%
\def\footnote{\protect\rmarkdownfootnote}
\IfFileExists{upquote.sty}{\usepackage{upquote}}{}

% pass through extra packages specified by user
\usepackage{bm}

%%%%%%%%%%%%%%%%%%%%%%%%%%%%%%%%%%%%%%%%%%%
%%%%%%%%%%%%%%%%%% EDIT TITLE %%%%%%%%%%%%%%%%%%
%%%%%%%%%%%%%%%%%%%%%%%%%%%%%%%%%%%%%%%%%%%

% change title format to be more compact
\usepackage{titling}

% create subtitle command for use in maketitle
\newcommand{\subtitle}[1]{
  \postauthor{
    \begin{center}\large#1\end{center}
    }
}

\setlength{\droptitle}{-1em}
\pretitle{\vspace{\droptitle}\centering\Huge}
\posttitle{\par\vskip 5.5em}

\title{
{\scshape\Large Department of Statistics 2019}\\
{\vskip 2.5em \scshape Predicting Community Engagement with Questions Across Online
Question-Answer Fora}\\
%{\includegraphics{lse.png}} % if you want to include LSE logo
}

\preauthor{\centering\LARGE}
\postauthor{\par\vskip 4em}

\author{Candidate Number: 10140}
\subtitle{\vspace{4em} Submitted for the Master of Science, London School of Economics,
University of London} % comment this out and you get *Missing \begin{document}*

\predate{\centering\Large}
\postdate{\par}

\date{\scshape August 2019}

\usepackage{color}
\usepackage[usenames,dvipsnames,svgnames,table]{xcolor}
\usepackage{hyperref}
\hypersetup{
     colorlinks = true,
     citecolor = gray
}

\usepackage{tocloft}

\renewcommand{\cftsubsecfont}{\normalfont\hypersetup{linkcolor=black}}
\renewcommand{\cftsubsecafterpnum}{\hypersetup{linkcolor=black}}

%%%%%%%%%%%%%%%%%%%%%%%%%%%%%%%%%%%%%%%%%%%
%%%%%%%%%%%%%%%% BEGIN DOCUMENT %%%%%%%%%%%%%%%%
%%%%%%%%%%%%%%%%%%%%%%%%%%%%%%%%%%%%%%%%%%%

\begin{document}

% Header and Footers
\pagestyle{fancy}
\chead{}
\rhead{}
\lfoot{}
\rfoot{} 
\lhead{}
%\rfoot{\footnotesize Page \thepage\ } % "e.g. Page 2"
\cfoot{\footnotesize \thepage\\}

% i, ii, iii etc. page numbering
\pagenumbering{roman}

\maketitle

\thispagestyle{empty}

\clearpage

\setcounter{page}{1}

% table of contents, list of figures and tables
\renewcommand{\contentsname}{Table of Contents}
\hypersetup{linkcolor=black}
\tableofcontents
\newpage
\hypersetup{linkcolor=black}
\listoffigures
\newpage
\hypersetup{linkcolor=black}
\listoftables
\hypersetup{linkcolor=black}
\newpage

\section*{Summary}

Formulating constructive questions and receiving answers to these
questions is crucial to how we examine, learn from and critically
analyse the world around us. The evolution of the world wide web and the
technologies that have emerged have given us an unprecendented ability
to engage with and learn from individuals around the world, and while
substantial attention has been dedicated to finding correct answers
(just ask Google), comparitively less has been devoted to how we can
improve the constructiveness of our questions. The domain of online
question-answer (Q\&A) communities is one setting where relevant and
well-researched questions is of particular importance, since domain
expert resources are generally scarce compared to the volume of new
questions (a problem known as information overload). This research
builds on the small amount of work aimed at questions in online Q\&A
communities, and begins to address the problem of information overload
by analysing a diverse range of questions from the StackExchange family
of Q\&A communities. Using only textual question content available at
the time questions are initially submitted, I construct models to
predict on the community-granted score for each question as a
measurement of community engagement. I find little to \textbf{no
improvement in prediction metrics after employing feature engineering
techniques sourced from the relevant literature across all fora.} Having
taken the step to predict a continuous proxy measurement of online
community engagement and begun \textbf{addressing temporal issues}, I
belive my \textbf{unique and original} research shows that there is
still much to be done to predict online community engagement objectively
and effectively, especially when considering the \textbf{chronological
nature of online Q\&A data}. Nevertheless, this research serves as a
stepping stone to accurately informing questioners of how their
questions will be received by an online community and potentially
nudging them to improve their questions before adding demand to these
communities, thereby improving the functioning and efficiency of these
platforms substantially. (317)

\clearpage

% 1, 2, 3 etc. page numbering
\pagenumbering{arabic}

\newpage

\section{\texorpdfstring{Introduction
\label{Intro}}{Introduction }}\label{introduction}

Modern interpersonal communication technologies made possible by the
internet have afforded us an exceptional level of connection and
engagement with the world. Billions of individuals now interact online
instantly, not only with people that they know, but with strangers
millions of miles away. One avenue of online interaction that has become
an extremely popular way in which users share knowledge about diverse
and nuanced subject matter is question-and-answer (Q\&A) websites such
as Yahoo! Answers, Quora, the StackExchange family and forums of Massive
Online Open Courses (MOOCs). These websites serve as dynamic, engaging
platforms where users seek answers to and discussions on complex and
technical questions that modern search engines are evidently yet unable
to fully address.

In these online Q\&A fora, producing relevant, well-researched and
high-quality questions is especially valuable not least since these
platforms suffer in particular from a low ratio of expert resources to
volume of new questions - a problem known as \emph{information overload}
({\textbf{???}}). The overarching hypothesis of this research is that if
questioners in online Q\&A fora were provided with information
specifically related to how well their questions will be received by
communities, then they could iterate to ``increase the signal'' of their
questions before exerting demand on community resources, thereby
mitigating the problem of information overload. This would no doubt
benefit questioners as they become better able to garner expert answers
to their improved questions, but also benefit entire communities as
overall functioning and efficiency is improved community-wide.

Addressing information overload in online Q\&A is a non-trivial problem
however, since providing predictions of \emph{community engagement} to
questioners in real time requires that only the information available
when new questions are formulated can be used as features, i.e.~question
content as opposed to other features such user characteristics, final
webpage viewing statistics etc. Furthermore, final predictions given to
questioners would also ideally be highly granular and direct questioners
towards how best to improve their questions (leading to an intersection
with the vast literature on recommendation systems), however the
undeniable first step, and what this research aims to achieve, would be
to ascertain if community engagement in online Q\&A fora can actually be
predicted with some measure of accuracy.

The broad research question for this paper can therefore be summarised
as the following:

\begin{center}
\emph{To what extent can community engagement with questions in online Q\&A communities be accurately predicted using only question content?}
\end{center}

While there is a substantial amount of literature that has addressed
Q\&A communities, interestingly the focus has been on identifying expert
users and high quality answers rather than looked at questions, despite
questions being the entry point for every interaction in communities. To
answer the research question above I analyse a diverse range of Q\&A
fora from the \href{https://stackexchange.com/sites\#}{StackExchange}
family of communities, and draw heavily on prior research done by Ravi
\emph{et al.} (2014) on question quality in online Q\&A fora.

Using only the textual content of questions, I predict on each
question's community-assigned \texttt{Score} - an aggregation of all
community \emph{up-votes} and \emph{down-votes} which I accept as an
objective and comprehensive metric for community engagement. In line
with the analysis of Ravi \emph{et al.} (2014), I employ Latent
Dirichlet Allocation (Blei, Ng and Jordan, 2003) to engineer latent
topic features from question content for predictions. I use elastic-net
regularised regression for the learning task and evaluate models using
root-mean-square error (RMSE).

It should be highlighted that the goal of this research is quantitative
prediction rather than qualitative, causal or inferential analysis.
While I will briefly touch on the differences between predictive model
results across the communities I analyse, I leave it to further research
to address more precisely the \emph{how} and \emph{why} of community
engagement in online Q\&A communities, rather than just the \emph{if}
that is explored here. To my knowledge this research is the first of its
kind to test latent topic models on a continuous and objective
measurement of online community engagement, analyse a diverse range of
communities, as well as have a real-life use-case, resulting in a
practical and unique contribution.

\textbf{My findings} show that there is still much work to be done to
accurately predict community engagement in online Q\&A fora. I find that
models that include features derived from the lengths and topics of
questions do \textbf{NOT} perform better than a baseline of just average
\texttt{Score} prediction from the training question set. I also find
that models across fora have \textbf{varying} levels of performance,
\textbf{providing evidence} against claims in Ravi \emph{et al.} (2014)
that topic models are applicable to different online Q\&A settings.

\textbf{Lastly, as a progressive} step forward in the research, I
evaluate the best performing model using a temporal train/test split,
taking into account the chronological nature of online Q\&A questions.
Here I find \textbf{almost no gain in predictive performance}, leading
me to believe that in order to accurately predict future community
engagement, models need to incorporate temporal and time-series
elements.

In the following section I discuss relevant literature in more detail.
This is followed by section \ref{Method} which discusses the data,
explores and validates my choice of \texttt{Score} as an objective
measurement of community engagement and describes the predictive model
used. Section \ref{Results} presents and discusses the results, section
\ref{Recom} makes some recommendations for areas of further research and
finally section \ref{Concl} makes some concluding remarks.

\newpage

\section{\texorpdfstring{Literature Review
\label{Lit}}{Literature Review }}\label{literature-review}

\subsection{Question-Answer
Communities}\label{question-answer-communities}

There is a substantial collection of research that has investigated
online Q\&A communities. Prior work has addressed answer quality (Jeon
\emph{et al.}, 2006; Shah and Pomerantz, 2010; Tian, Zhang and Li,
2013), satisfaction of questioners (Liu, Bian and Agichtein, 2008) and
the behaviour of highly productive, expert community members (Riahi
\emph{et al.}, 2012; Sung, Lee and Lee, 2013). Two common frameworks for
prior work has been the optimisation of routing questions to experts (Li
and King, 2010; Li, King and Lyu, 2011; Zhou, Lyu and King, 2012; Shah
\emph{et al.}, 2018), and matching questions in accordance with answerer
interest in the form of a recommendation system (Wu, Wang and Cheng,
2008; Qu \emph{et al.}, 2009; Szpektor, Maarek and Pelleg, 2013).

This research differs from this previous work on Q\&A fora in two
respects. Firstly, I focus on questions rather than user or answer
characteristics, not only because they have received substantially less
attention in the literature, but because it has been shown that question
quality can substantially impact the quality of answers (Agichtein
\emph{et al.}, 2008). Questions in online Q\&A fora are also the initial
event that all community engagement follows from and thus maximising
positive community engagement with questions will almost certainly
improve the evolution and functioning of communities.

The second distinction from prior research is the framework in which
this research is placed. I choose a framework of community engagement
and interaction with user actions rather than the systems-based
optimisation of question-answer routing and matching and instead
concentrate on how questioners can be nudged to improve the content of
their questions before encumbering community resources.

\textbf{Community engagement is a rather broad term in literature
ranging across fields and disciplines, but I have not found any
literature relating to community engagement in the context of online
Q\&A fora.}

With the promise of this real-life application which significantly
benefit both questioners and communtities, it remains to be seen if
community engagement in online Q\&A fora can be successfully predicted.
Owing to a large overlap between this goal and the literature on
predicting question quality in online Q\&A fora, I discuss this
literature next.

\subsection{Question Quality}\label{question-quality}

\textbf{LOTS OF WORK}

High-quality questions assuredly lead to positive community engagement,
however \textbf{the only difference may be the specific aspects of
question content that communities value across communities.} Thus, while
the literature discussed here refer to measuring and predicting
``question quality'', I assert that ``community engagement'' is a more
robust interpretation of what they are measuring and so for the sake of
discussion I will refer to question quality as well.

Recent work has looking at predicting question quality for the large
Q\&A community \href{http://answers.yahoo.com}{Yahoo! Answers}
(Agichtein \emph{et al.}, 2008; Bian \emph{et al.}, 2009; Li \emph{et
al.}, 2012), but while this dataset has metrics for assessing answer
quality in the form of answer ``up-votes'', it lacks a similarly
community-attributed and objective metric for question quality.
Agichtein \emph{et al.} (2008) thus define question quality using
question semantic features (lexical complexity, punctuation, typos
etc.), Bian \emph{et al.} (2009) manually label 250 questions and Li
\emph{et al.} (2012) combine the number of answers, number of tags, time
until the first answer, author judgement and domain expertise to
construct their ground truth.

Fortunately, my datasets are from the StackExchange family of Q\&A fora
which are rich in community engagement variables like question
up-/down-votes and view-counts. Coming directly from the data, these
metrics are objective rather than human-labelled and are also therefore
not limited in terms of samples from the data (we can use the whole
dataset).

The predictive models employed in the question quality literature have
also evolved substantially. Previous work has modelled question quality
based on the reputation of the questioner, question categories and
lexical characteristics of questions (length, misspelling, words per
sentence etc.) (Agichtein \emph{et al.}, 2008; Bian \emph{et al.}, 2009;
Anderson \emph{et al.}, 2012; Li \emph{et al.}, 2012).

A fundamental distinction is that I use only the features available at
the time a question is initially asked which is congruent with the goal
of being able to provide real-time information to questioners before
they submit questions to a community. I also don't use any features
derived from user attributes, since doing otherwise would not work well
for questions asked by new users.

\newpage

\subsection{\texorpdfstring{Ravi \emph{et al.}
(2014)}{Ravi et al. (2014)}}\label{ravi2014}

\textbf{LOTS OF WORK}

A paper that made much headway in the classification and prediction of
what they assume is question quality is Ravi \emph{et al.} (2014), who
use the largest and oldest StackExchange site,
\href{https://stackoverflow.com}{StackOverflow} I mirror much of the
analysis in Ravi \emph{et al.} (2014), however I believe I build and
diverge from their analysis significantly in a number of ways.

As discussed, much of the literature is oriented towards ``question
quality'' and Ravi \emph{et al.} (2014) decide to incorporate a
question's \texttt{Score} into their ground truth for question quality,
yet I posit that what these studies are measuring is instead more
accurately characterised as community engagement. My opinion is that
question quality is much more nuanced than prior research has asserted,
i.e.~while most communities will value universal aspects of questions
like legibility, coherence, relevance and prior-research, it is
difficult to accurately define how much of this contributes to a
universal inherent ``quality'' objectively compared to
community-specific traits that communities will naturally value
(i.e.~closed-end questions in the natural sciences, discussion-promoting
for social sciences). Thus while I also incorporate the \texttt{Score}
metric into a response variable, my characterisation of this ground
truth as community engagement is broader and more inclusive.

Another departure from the analysis in Ravi \emph{et al.} (2014) that I
make, is I consider a far more diverse range of communities to compare
how models perform across fora. Ravi \emph{et al.} (2014) specifically
state that ``{[}their{]} methods do not rely on domain-specific
knowledge'' and therefore ``{[}they{]} believe {[}the methods{]} are
applicable to other CQA settings as well''. I believe that community
behavior is too diverse to be universally predicted by a single model,
\textbf{thus this will be interesting to test in the results.}

A last distinction between my analysis and Ravi \emph{et al.} (2014) is
that they treat the research aim as a classification problem, quite
arbitrarily defining a threshold for their response variable to
distinguish between ``good'' and ``bad'' questions. Despite
\textbf{making it a more complex problem}, I opt to predict on a
continuous response since that would provide a better indication to
users of how well it is predicted that a community will react to their
question.

Ravi \emph{et al.} (2014) manage impressive results however: using
textual features and latent topics extracted from question content
(i.e.~question \texttt{Title} and \texttt{Body}), their predictions on
\texttt{Score}/\texttt{ViewCount} yield accuracy levels of 72\% on their
StackOverflow dataset. I will be emulating this part of their research,
and thus a discussion of topic modelling is necessary.

\subsection{\texorpdfstring{Topic Modeling
\label{model_lit}}{Topic Modeling }}\label{topic-modeling}

\textbf{LOTS OF WORK}

Bayesian models have recently achieved immense popularity to solve a
diverse range of structured prediction challenges in Natural Language
Processing (NLP) (Chiang \emph{et al.}, 2010). Blei, Ng and Jordan
(2003) presented Latent Dirichlet Allocation (LDA) topic models as
generative Bayesian models for documents to uncover hidden topics as
probability distributions over words. LDA can therefore be useful in
unearthing underlying semantic structures of documents and to infer
topics of the documents.

Attaining accuracy scores of up to 72\% for ``good'' and ``bad''
questions, Ravi \emph{et al.} (2014) have indeed shown the capability of
using latent topics derived from LDA modeling. Ravi \emph{et al.}
(2014)'s final predictive model is based on work by Allamanis and Sutton
(2013), who also analysed the StackOverflow dataset, but did not look at
\textbf{any form} of ``question quality''. They uses LDA models at three
levels: across the whole question body, on code chunks in the question
body, and on the question body without noun phrases.

Ravi \emph{et al.} (2014) choose to model latent topics 1) Globally in
order to capture topics over questions as a whole, 2) locally to seize
sentence-level topics, and finally use a Mallows model (Fligner and
Verducci, 1986) for a global topic structure to administer structural
constraints on sentence topics in all questions.

Since Ravi \emph{et al.} (2014) see no substantial gains in predictive
accuracy using the Mallows model, I only employ the LDA features
(\textbf{and maybe word-embedding features}), I also split my train/test
temporally and differ in deleting low viewcount questions. Despite
mirroring the methodology in Ravi \emph{et al.} (2014), I critique and
build on it in many ways and will begin this discussion on my
methodology now.

\newpage

\section{\texorpdfstring{Methodology
\label{Method}}{Methodology }}\label{methodology}

\subsection{\texorpdfstring{Data \label{Data}}{Data }}\label{data}

The \href{https://stackexchange.com/sites\#traffic}{StackExchange}
family of online Q\&A fora are a diverse range of over 170 community
websites covering topics from vegetarianism to quantum computing to
bicycles. Over and above the textual content of all questions, answers
and comments posted since each communities conception, rich meta-data on
all communities is publicly available in XML files compressed in 7-Zip
format at \href{http://archive.org/download/stackexchange}{archive.org}.

The \textbf{five} StackExchange datasets that I chose to analyse are
displayed in table \ref{tab:fora}, along with a short description.

\footnotesize

\begin{longtable} {@{} cccccp{7cm} @{}}
\caption{\textbf{Details of Datasets}}
\label{tab:fora}\\ 
\toprule
Forum & Questions & Answers & Users & Site Age & Description \\ 
\midrule
Stack Overflow & 18m & 28m & 11m & 11yrs & Q\&A for professional and enthusiast programmers \\
Super User & 420k & 605k & 795k & 10yrs & Q\&A for computer enthusiasts and power users \\ 
Math & 1.1m & 1.6m & 567k & 9yrs & Q\&A for people studying math at any level \\
Cross Validated (Stats) & 143k & 143k & 209k & 9yrs & Q\&A for people interested in statistics \\ 
English & 107k & 250k & 267k & 9yrs & Q\&A for English language enthusiasts \\ 

%Fitness & 21k & 8.2k & 16k & Q\&A for athletes, trainers and physical fitness professionals & \\ 
%Economics & 17k & 7.8k & 9.9k & For those studying, teaching, researching and applying economics/econometrics & \\
%Buddhism & 9.7k & 5.8k & 19k & Discussions on Buddhist philosophy, teaching and practice & \\
%Health & 13k & 5.6k & 4.5k & For professionals in the medical and allied health fields & \\ 
%Interpersonal & 22k & 3.1k & 13k & Q\&A for anyone wanting to improve their interpersonal skills & \\
\bottomrule
\end{longtable}

\normalsize

Owing to the size of the datasets, I process and analyse the data with
PySpark, a Python API for the open-source cluster-computing framework
\href{http://spark.apache.org}{Apache Spark}. In the interests of
transparency and reproducibility, the entire PySpark codebase for the
processing and modelling of the data done can be found at
\url{https://github.com/BCallumCarr/msc-lse-thesis/}.

The data from the five selected fora is downloaded, decompressed and
converted to \href{https://parquet.apache.org}{Parquet} format. The
following variables are of interest to my analysis from the data:

\setstretch{0.95}

\begin{itemize}
\item
  \texttt{Score}: The difference between registered-user granted
  up-votes and down-votes for a question
\item
  \texttt{ViewCount}: A counter for the number of page views a question
  receives (form both registered and non-registered users)
\item
  \texttt{Title}: The text of the question title
\item
  \texttt{Body}: The text of the question body
\item
  \texttt{CreationDate}: A datetime variable indicating when the
  question was initially posted
\item
  \texttt{AnswerCount}: Number of answers a question receives (questions
  only)
\item
  \texttt{CommentCount}: Number of comments a post receives
\item
  \texttt{FavoriteCount}: Number of times users favourite a question
  (questions only)
\item
  \texttt{AcceptedAnswerId}: Indicates which answer the question-asker
  selects as accepted (questions only)
\item
  \texttt{ClosedDate}: A date variable indicating if a question was
  closed (questions only)
\end{itemize}

\setstretch{1.25}

There are two options for selecting the final data I wish to analyse:
Selecting an equal number of questions from communities with the
datasets spanning different lengths of time, or selecting a common
length of time but then having datasets with different sizes. This is
our first insight into how the temporal nature of online Q\&A data can
complicate analyses. Since the main goal of this research is not to
compare communities (although a natural comparison will evidently
surface), I choose to mitigate as much temporal nature in the datasets
by choosing a relatively short and uniform time period in which to
extract the final data. The combination of analysing data over a uniform
time period, and also trimming away recent questions that would not have
existed in communities long enough to garner sufficient votes and views
results in a final dataset that should have little to no temporal nature
and bias.

The dates I choose to examine the datasets are from the \textbf{1st of
September 2010 to the 1st of September 2011}, since this is from the
start date of Maths, Stats and English. This trims the initial number of
questions down to \textbf{1 113 802} questions, or more specifically 1
042 477 for StackOverflow, 40 589 for SuperUser, 18 131 for Math, 8 537
for English and 4 068 for Stats.

It should be noted that the variation in the length of time periods from
which questions were extracted across fora may complicate comparison
between fora if the data exhibit temporal effects and trends - this will
be explored in more detail a bit later. I now move onto formalising the
measurement of community engagement that will be predicted on.

\subsection{A Measurement of Community
Engagement}\label{a-measurement-of-community-engagement}

\subsubsection{\texorpdfstring{An Exploration of Community Engagement
Variables
\label{Vars}}{An Exploration of Community Engagement Variables }}\label{an-exploration-of-community-engagement-variables}

There are a number of ways that online Q\&A community members
interact\ldots{}

One aspect of this research that stands out as an area for further
research is the fact that only one target variable, the question
\texttt{Score}, was considered as a measurement of community engagement,
whereas in reality there are others already available in the data. There
are metrics recording how many interactions a question receives, such as
\texttt{AnswerCount} (the number of answers for a question) and
\texttt{CommentCount} (the number of comments for a question), which all
signify at least some engagement with a question, although whether this
is positive or negative engagement is unknown. In response to this, one
could construct a variable relating to the linguistic sentiment of the
answers (not comments, since comments need not be directed at the
original questioner), however the subtleties of identifying sarcastic
and condescending answers and comments might be overly difficult,
especially since communities would value pleasant critical feedback.

Another variable that is a direct indication of questioner satisfaction
is whether they deem an answer to have successfully addressed their
question, which is recording in the variable \texttt{AcceptedAnswer}.
This variable is not without its own issues, since users may find
utility from multiple answers and neglect to formally select an accepted
answer at all, biasing the number of formally solved questions downwards
and confounding the response variable. Furthermore, answers are commonly
posted as comments and vice-versa (see
\url{https://meta.stackexchange.com/questions/17447/answer-or-comment-whats-the-etiquette}),
and this too would confound the predictive results for this variable.
Comments being posted as answers (i.e. ``clogging up'' the list of
answers), can be a case of users who don't have the required level of
reputation to comment yet or a case of users chasing reputation points
by using jokes, which obscurs the reputation measurement as users get
voted up for being humourous rather than their expertise. Treating this
variable as the target variable also situates the research problem in
terms of exclusive utility to the user, whereas the \texttt{Score}
variable is a more broader measurement of how the community values
questions, which in turn should translate into utility for the
questioner. One assumption that would mitigate issues surrounding the
\texttt{AcceptedAnswer} variable would state that the discussed
anomalies are not common enough and are not biased to specific posts
with an even and randomly distribution over the data it would not
significantly effect the results.

One last response variable for consideration is the number of times a
post is edited, the \texttt{EditCount}. This variable could have two
implications however - more edits signify more effort needed to bring
the question in the desired state (i.e.~it is inversely proportional to
positive community engagement), or more edits signify more energy
willingly devoted to improving the question because it will add value to
the community (and thus it is directly proportional to positive
community engagement).

**I won't use the \texttt{AcceptedAnswer} variable since I believe it
unreliable, and providing less detail as a binary variable than
continuous variables like the \texttt{Score} and \texttt{ViewCount}.

\subsubsection{\texorpdfstring{An Exploration of Community Engagement
Variables
\label{Vars}}{An Exploration of Community Engagement Variables }}\label{an-exploration-of-community-engagement-variables-1}

The \texttt{Score} and \texttt{ViewCount} variables are the best
candidates for community engagement and thus we perform exploratory
analysis on them:

\footnotesize

\begin{longtable} {@{} lccccc @{}}
\caption{\textbf{Descriptives for the ViewCount Variable}}
\label{tab:bestworst}\\ 
\toprule
\textbf{Forum} & \textbf{Count} & \textbf{Mean} & \textbf{SD} & \textbf{Min} & \textbf{Max} \\ 
\midrule
Buddhism      &   3120 &   316.16 &   737.08 &   10 &  21498 \\
Economics     &   3120 &   178.24 &   646.26 &    2 &  14055 \\
Fitness       &   3120 &   534.35 &  2426.63 &    8 &  78829 \\
Health        &   3120 &   240.40 &  1128.95 &    2 &  23098 \\
Interpersonal &   3120 &  4520.74 &  7492.72 &    6 &  79049 \\
\bottomrule
\end{longtable}\begin{center} Source: Own calculations in PySpark.\end{center}

\normalsize

\footnotesize

\begin{longtable} {@{} lccccc @{}}
\caption{\textbf{Descriptives for the Score Variable}}
\label{tab:bestworst}\\ 
\toprule
\textbf{Forum} & \textbf{Count} & \textbf{Mean} & \textbf{SD} & \textbf{Min} & \textbf{Max} \\ 
\midrule
Buddhism      &   3120 &   2.01 &    2.19 &   -7 &   24 \\
Economics     &   3120 &   1.47 &    2.70 &   -7 &   61 \\
Fitness       &   3120 &   1.77 &    2.14 &   -6 &   28 \\
Health        &   3120 &   2.05 &    2.06 &   -5 &   27 \\
Interpersonal &   3120 &  16.38 &   23.70 &   -9 &  265 \\
\bottomrule
\end{longtable}\begin{center} Source: Own calculations in PySpark.\end{center}

\normalsize

From the above descriptive tables we seee that there is much
heterogeneity across the \texttt{ViewCount} and \texttt{Score}
variables. Most notably, the averages and variances of
\texttt{ViewCount} and \texttt{Score} from Interpersonal are orders of
magnitude higher than other fora. This shows begins to shed light on how
distinctly these communities appear to operate quite distinctly, either
due to community behaviour or questioner behaviour - not community size
because as we see in table \ref{tab:fora}, this appears to not be
related to the \textbf{descriptive statistics above.} \textbf{As feels
trivially true, if communities value certain aspects of questions
differently or behave differently, predicting community engagement with
a universal model feels quite the challenging.} For further information
on these variables, we plot density plots of the \texttt{Score} and
\texttt{ViewCount} variables:

\footnotesize

\begin{figure}
\caption{\textbf{Density Plots}}
\label{fig:density}

\begin{center}\includegraphics[width=0.8\linewidth]{../../01-python-code/00-workspace/01-eda/01-graphs/viewcount-density-plot} \end{center}



\begin{center}\includegraphics[width=0.8\linewidth]{../../01-python-code/00-workspace/01-eda/01-graphs/score-density-plot} \end{center}
\centering {\footnotesize Source: Own calculations in PySpark.}
\end{figure}

\normalsize

In both density plots we see much visual confirmation of the differences
between fora, most notably the \textbf{outlier-ish data points for both
the \texttt{Score} and \texttt{ViewCount} variables from the
Interpersonal forum.} The distributions of both variables across all
fora also appear to be significantly uneven and negatively skewed. This
seems to coincide with Benford's, Zipf's and Pareto's laws
(\url{https://workplace.meta.stackexchange.com/questions/5018/massive-viewcount-difference}).

There is also the matter of the websites functionings themselves.
Although questions on all StackExchange sites are open to the public,
posting a question in a community requires registration with an email
address and a username and once registered, users start with a
\emph{reputation} level of 1
(\url{https://meta.stackexchange.com/questions/7237/how-does-reputation-work}).
Key reputation levels include:

\setstretch{0.65}

\begin{itemize}
\item
  15: Users are allowed to ``up-vote'' questions and answers
\item
  125: Users can ``down-vote'' questions and answers
\item
  1000: Users can edit any question or answer.
\end{itemize}

\setstretch{1.25}

At least for the \texttt{Score} variable, the contrasting reputation
levels for up- and down-voting privileges (15 and 125 respectively)
would bias this variable to be negatively skewed, as seen in figure
\ref{fig:density}, making it more likely that questions will have a
higher positive \texttt{Score} rather than a negative one.

Asides from the heterogeneity across fora, it appears that both
variables exhibit similar distributions, and this leads us to discuss
and identify what they represent, and how they could be used to measure
community engagement.

This leaves us with the \texttt{Score} and \texttt{ViewCount} variables
for analysis.

\newpage

\subsubsection{\texorpdfstring{\texttt{Score} versus \texttt{ViewCount}
for Measuring Community
Engagement}{Score versus ViewCount for Measuring Community Engagement}}\label{score-versus-viewcount-for-measuring-community-engagement}

With our two candidates for measuring community engagement, we now move
onto a deeper discussion of how these variables arise. Only members that
have registered with the community are able to up-vote and down-vote and
thus contribute to the \texttt{Score}, but owing to all questions being
open to the public, \texttt{ViewCount} variable registers views from 1)
registered users that can vote, 2) registered users that can't vote due
to a reputation level below 15, and 3) non-registered members.

This caveat influences the methodology of Ravi \emph{et al.} (2014) in
their goal of predicting ``question quality'' heavily. They decide to
use a composite response variable, \texttt{Score}/\texttt{ViewCount}
stating that considering \texttt{Score} alone might lead to conflating
popularity with question quality, since a higher \texttt{ViewCount}
would definitely be linked to a higher \texttt{Score} variable upwards
owing to the unsymmetrical reputation privileges. This is easily seen
when looking at a table of correlations between the \texttt{Score} and
\texttt{ViewCount} variables across fora in table \ref{tab:corr} below:

\footnotesize

\begin{longtable} {@{} cc @{}}
\caption{\textbf{Score and ViewCount Correlations Across Fora}}
\label{tab:corr}\\ 
\toprule
\textbf{Forum} & \textbf{Correlation} \\ 
\midrule
Buddhism & 0.41 \\
Economics & 0.67 \\
Fitness & 0.26 \\
Health & 0.21 \\
Interpersonal & 0.87 \\
\bottomrule
\end{longtable}\begin{center} Source: Own calculations in PySpark.\end{center}

\normalsize

In the table above we see correlations ranging from 0.21 for Health, to
0.87 for Interpersonal. Thus it appears that the \texttt{Score} and
\texttt{ViewCount} are indeed inexorably linked, but \textbf{we can dive
deeper into what they actually represent}. The framework I would like to
develop is that of \emph{within-community} engagement versus
\emph{outer-community engagement}. I assert that voting by community
members, and consequently the \texttt{Score} variable, is purely a
within-community metric because users are required to commit and
register with a community to contribute to this variable.
\texttt{ViewCount} on the other hand, can be defined as both a within-
and outer- community engagement variable, since it does not distinguish
voting or non-voting status when registering question views.

I decide to mix focus solely on within-community engagement in my
methodology and analysis, and therefore use on the \texttt{Score}
variable as the response to be predicted. While this may imply that what
I am predicting is also \textbf{popularity} of questions, I believe that
popularity is in itself a measurement of community engagement. I further
believe that in providing predictive information to questioners about
their new questions, questioners would be more interested in seeing
their final \texttt{Score} prediction rather than \texttt{ViewCount} or
\texttt{Score}/\texttt{ViewCount}.

\subsubsection{A Final Response
Variable}\label{a-final-response-variable}

Finally, table \ref{tab:bestworst} displays the titles of a selection of
community questions with the highest and lowest \texttt{Scores}, i.e.~a
selection of the ``best'' and ``worst'' questions according to the
methodology I have chosen.

\footnotesize

\begin{longtable} {@{} cccp{11cm} @{}}
\caption{\textbf{Highest and Lowest Scored Questions Across Fora}}
\label{tab:bestworst}\\ 
\toprule
\textbf{Forum} & \textbf{Score} & \textbf{ViewCount} & \textbf{Title} \\ 
\midrule
Buddhism &     24 &       7228 &  Is low self-esteem a Western phenomenon? \\
Buddhism &     -7 &        103 &                 Who remembers the Buddha? \\
Buddhism &     -7 &        447 &                Why are buddhists hostile? \\
\hline
Economics &     61 &      14055 &  What are some results in Economics that are both a consensus and far from common sense? \\
Economics &     -7 &        179 &                                                              What is feminist economics? \\
\hline
Fitness &     28 &      12376 &  Why does one person have lots of stamina and another doesn't? \\
Fitness &     -6 &         54 &                            Gaining fat for muscles-stomach fat \\
\hline
Health &     27 &       4364 &  What are known health effects of smoking e-cigarettes \\
Health &     -5 &         35 &  Do "whole body jolts" experienced from things like tasting vinegar, a puppy licking one's hear, chalk screeching, etc. reach the median nerve? \\
\hline
Interpersonal &    265 &      32147 &                     What to do if you are accidentally following someone? \\
Interpersonal &     -9 &       1327 &  How can I tell if family members consider my unvaccinated kids a threat? \\
Interpersonal &     -9 &        937 &             How to tell employees that I don't mean my insults seriously? \\
\bottomrule
\end{longtable}\begin{center} Source: Own calculations in PySpark.\end{center}

\normalsize

It appears that questions that are well-received by a community are
genuine and discussion-promoting, whereas those received negatively by
communities, have elements of sarcasm and insincerity in the sense that
that are not actually looking for answers - i.e. \textbf{\ldots{}}.
Social norms also appear to be prevalent in community reactions to
questions - case in point being the unvaccinated kids question from
Interpersonal.

\subsubsection{\texorpdfstring{Potential Methodological Issues
\label{Issues}}{Potential Methodological Issues }}\label{potential-methodological-issues}

One possible confounding factor for the response variable that is worth
considering is that questions can be edited, not only by the original
poster, but by anyone with a level of reputation of 1000 or more.
General cross-community guidelines for editing include addressing
grammar and spelling issues, clarifying concepts, correcting minor
mistakes, and adding related resources and links. The concern here is
that users could vote, comment and answer on substantially different
questions over time as a question is edited from it's original form.
\textbf{The simplifying assumption that I make here is that most edits,
if any at all, would happen quickly as moderators are made aware of
offending questions and thus the majority of views and votes would
happen on final, edited questions. I therefore choose final edited
question content to predict on.}

Another factor is community behaviour confusion - there seems to be a
less-than-full consensus of when exactly to up- or
down-vote\footnote{https://meta.stackexchange.com/questions/12772/should-i-upvote-bad-questions}
despite general guidelines on StackExchange sites stating that up-votes
should be given if a question shows prior research, is clear and useful,
and down-voting the opposite.

A second methodological adjustment that Ravi \emph{et al.} (2014) make
with their data is to only consider questions above a certain minimum
\texttt{ViewCount} threshold. Their reasoning behind this is so that
they can be more confident of the final dataset containing questions
that have been viewed by qualifying users that can vote, or in other
words their claim is that questions with higher \texttt{ViewCounts} have
a higher probability of having been seen by community members able to
vote.

I believe this is a false claim, since one could just as easily argue
that new questions that begin with a low \texttt{ViewCount} are more
likely to see engagement from proactive community members, especially if
these questions doesn't generate enough webpage activity to rise as the
top hit for search engines (which would lead to more non-community
member activity contribution to views). Since there is additionally no
data on the distribution of qualifying and non-qualifying user
contributions to the \texttt{ViewCount} variable, therefore I opt to not
disregard any questions below a certain \texttt{ViewCount} threshold.

\newpage

\subsection{\texorpdfstring{Model \label{Model}}{Model }}\label{model}

\subsubsection{Train/Test Split}\label{traintest-split}

Let \(q_i\) denote question \(i\) out of all questions \(Q\) for a given
forum. I split the datasets into a training set \(Q_\text{train}\)
(50\%) and a testing set \(Q_\text{test}\) (50\%), each with 1 560
questions. \textbf{I choose a 50/50 train/test split because I believe
that the size of the datasets allows for enough training data.} The
standard deviations of a random splitting of training and testing sets
is displayed in table \ref{tab:rand_tr_te} below.

\textbf{Use SD or \(\sigma\)?}

\footnotesize

\begin{longtable}[htbp] {@{} lccc @{}} 
\caption{\textbf{Random Train/Test Split Standard Deviations}} 
\label{tab:rand_tr_te} \\
\toprule
\textbf{Forum} &  \textbf{Train SD} &  \textbf{Test SD} & \text{\% Difference} \\
\midrule
Buddhism & 2.1 & 2.27 & 8.1 \\
Economics & 1.86 & 3.34 & 79.57 \\
Fitness & 2.18 & 2.1 & -3.67 \\
Health & 2.11 & 2.01 & -4.74 \\
Interpersonal & 22.37 & 24.96 & 11.58 \\
\bottomrule
\end{longtable}\begin{center} Source: Own calculations in PySpark\end{center}

\normalsize

We see in table \ref{tab:rand_tr_te} that the standard deviations of
\textbf{one forum} are clearly distinct from the others.

However, when the train/test split is done so that the training set
questions chronologically precede the testing set questions, there are
substantial differences in standard deviations, as shown in table
\ref{tab:time_tr_te}.

\footnotesize

\begin{longtable}[htbp] {@{} lccc @{}} 
\caption{\textbf{Temporal Train/Test Split Standard Deviations}} 
\label{tab:time_tr_te} \\
\toprule
\textbf{Forum} &  \textbf{Train SD} &  \textbf{Test SD} & \text{\% Difference} \\
\midrule
Buddhism & 2.42 & 1.82 & -24.79 \\
Economics & 3.16 & 2.1 & -33.54 \\
Fitness & 2.17 & 2.09 & -3.69 \\
Health & 1.96 & 2.16 & 10.2 \\
Interpersonal & 25.97 & 20.47 & -21.18 \\
\bottomrule
\end{longtable}\begin{center} Source: Own calculations in PySpark\end{center}

\normalsize

This shows that the data is heterogenous with regard to time, either due
to how the communities have evolved over time or how questions have
evolved.

I use the random train/test split for the first part of the analysis and
the temporal split for the second. This touches on a point that was not
considered in Ravi \emph{et al.} (2014) nor in previous research to my
knowledge - the temporal nature of online Q\&A questions. I believe that
predicting \texttt{Scores} of future questions may prove a substantially
more difficult task than just randomising the training and testing
question sets.

Note however that not included a temporal element to my model, so if
there are some time-series trends in the data (to do with the struture
of the websites changing etc.), then the temporal prediction will be
poor.

\textbf{Since this analysis has already taken the first step from prior
research to look more broadly at community engagement, and a continuous
response in addition, potential temporality of the data will not be
thoroughly explored in the form of implementation of time-series models,
but will be touched on by comparing model results for a random
train/test split versus chronological.}

\subsubsection{Elastic-net Regularised Regression
Model}\label{elastic-net-regularised-regression-model}

I use elastic-net regularised regression to predict the score, denoted
\(s_i\), of each question using only features derived from the raw
textual \texttt{Body} and \texttt{Title} independent variables, which I
shall denote \(\bm{x'}_i\). The learning objective can therefore be
summarised as finding a coefficient vector \(\bm{\beta}\) which
minimises the Root Mean Squared Error:

\begin{align} \label{eq:rmse}
\underset{\bm{\beta}}{\text{minimise}} \quad \sqrt{ \frac{1}{|Q_\text{train}|} \sum_{ q_{i} \in Q_{\text{train}} } ( s_i - {\bm{\beta}\bm{x'}_i} )^2 + \Psi }
\end{align}

where

\begin{align} \label{eq:penalty}
\Psi = \lambda \sum_{j=1}^p ( \alpha\beta_j^2 + (1-\alpha)|\beta_j| )
\end{align}

is the elastic net penalty term. \textbf{WHY CHOSE RMSE AS METRIC}

In this term, \(\lambda\) is the regularisation parameter, \(\alpha\) is
a weighting coefficient for the \(L_1\) and \(L_2\) norms of the input
variables, corresponding to the lasso and ridge penalties respectively.
\textbf{MORE}

\textbf{LASSO better when there are variables that are useless (they get
shrunken to 0), RIDGE better when all are useful because it will shrink
parameters but not eliminate.}

I use 2-fold cross validation - 2 because increasing the number of folds
did not lead to large gains in RMSE reduction over models in general,
and also drastically increased computation time.

\subsubsection{Question Content}\label{question-content}

A number of preprocessing steps are applied to the \texttt{Body} and
\texttt{Title} to obtain the final features \(\bm{x'}_i\) that are
discussed subsequently - I parsed the HTML of the question content in
the \texttt{Body} variable, tokenised (with punctuation) both the
\texttt{Body} and \texttt{Title} texts, removed English stopwords and
stemmed tokens using Porter-stemming ({\textbf{???}}).

I first extract features relating to the length of questions'
\texttt{Body} and \texttt{Title}, i.e.~token count, sentence count and
character count. Then, the actual unigram text of the question
\texttt{Body} and \texttt{Title} are used as features in the form of
term frequency -- inverse document frequencies (TF-IDF). \textbf{MORE}
\textbf{Since Ravi \emph{et al.} (2014) do not use higher order ngrams,
I also stick to unigrams, resulting in quick and compact learning.}

\subsubsection{Topic Modelling}\label{topic-modelling}

I train an LDA model globally over all questions in \(Q\). I use the
online LDA learning framework in the Pyspark \texttt{pyspark.sql.ml}
package to generate topic distributions over words for each question and
add these as model features. This results in features made up of weights
\(\theta_{qt}\) for a topic \(t\) in a question \(q\), and
\(\theta_{qt}=P(t|q).\)

I choose \(K=10\) topics

\textbf{Online LDA works like this}

\newpage

\section{\texorpdfstring{Results
\label{Results}}{Results }}\label{results}

\subsection{Random Train/Test Split}\label{random-traintest-split}

To establish a baseline for the predictive performance of the models,
table \ref{tab:rand_mean_model} displays training and testing RMSE
values across fora for a model that predicts the constant mean of the
training set for every question in the testing set.

\footnotesize

\begin{longtable}[htbp] {@{} lccc @{}} 
\caption{\textbf{Constant Mean Model}} 
\label{tab:rand_mean_model} \\
\toprule
\textbf{Forum} &  \textbf{Train RMSE} &  \textbf{Test RMSE} &  \textbf{Time (s)} \\
\midrule
Buddhism      &                  2.10 &               2.27 &                0.50 \\
Economics     &                  1.86 &               3.34 &                0.58 \\
Fitness       &                  2.18 &               2.10 &                0.52 \\
Health        &                  2.11 &               2.01 &                0.49 \\
Interpersonal &                 22.36 &              24.96 &                0.58 \\
\bottomrule
\end{longtable}\begin{center} Source: Own calculations in PySpark\end{center}

\normalsize

The values in the Test RMSE column in table \ref{tab:rand_mean_model}
are considered the low benchmarks that future models must improve upon.
Interestingly, test RMSE is lower for all communities than train RMSE,
thus it appears that there is substantially more noise in the training
sets (remember that the training sets are older questions as well).

\textbf{The RMSE is different per community owing to the different
standard deviations of the data as a whole seen in the EDA\ldots{}}

\footnotesize

\begin{longtable}[htbp] {@{} lcccc @{}} 
\caption{\textbf{ViewCount Model}} 
\label{tab:rand_view_model} \\
\toprule
\textbf{Forum} &  \textbf{Train RMSE} &  \textbf{Test RMSE} &  \textbf{Test Gain (\%)} &  \textbf{Time (s)} \\
\midrule
Buddhism      &                 1.95 &              2.03 &              10.57 &               7.07 \\
Economics     &                 1.70 &              2.55 &              23.65 &               4.36 \\
Fitness       &                 2.06 &              2.13 &              -1.43 &               5.70 \\
Health        &                 2.06 &              1.98 &               1.49 &               5.61 \\
Interpersonal &                10.98 &             12.72 &              49.04 &               5.94 \\
\bottomrule
\end{longtable}\begin{center} Source: Own calculations in PySpark\end{center}

\normalsize

Table \ref{tab:rand_view_model} is the high benchmark owing to the
strong correlations seen in table \ref{tab:corr}. It is also vacuous,
since the final \texttt{ViewCount} of a question is not available for
new questions.

\footnotesize

\begin{longtable}[htbp] {@{} lcccccc @{}} 
\caption{\textbf{Length Model}} 
\label{tab:rand_count_model} \\
\toprule
\textbf{Forum} &  \textbf{Train RMSE} &  \textbf{Test RMSE} &  \textbf{Test Gain (\%)} &  \textbf{Time (s)} & \textbf{Elastic Param} &  \textbf{Reg'tion Param} \\
\midrule
Buddhism      &              2.08 &           2.25 &            0.88 &           14.04 &              0.01 &              0.01 \\
Economics     &              1.86 &           3.33 &            0.30 &           10.34 &              0.01 &              1.00 \\
Fitness       &              2.17 &           2.09 &            0.48 &            7.60 &              0.01 &              1.00 \\
Health        &              2.09 &           2.01 &           -0.00 &            7.93 &              1.00 &              0.01 \\
Interpersonal &             22.31 &          24.88 &            0.32 &           14.51 &              1.00 &              1.00 \\
\bottomrule
\end{longtable}\begin{center} Source: Own calculations in PySpark\end{center}

\normalsize

Table \ref{tab:rand_count_model} shows mild gains from approximately
0.3\% above the constant mean benchmark in the Test Gain column. We also
see that the Interpersonal forum differs from the others in that the
grid search found a regularisation parameter of 1 to be the most
optimal, implying \ldots{}

\footnotesize

\begin{longtable}[htbp] {@{} lcccccc @{}} 
\caption{\textbf{Unigram Textual Model}} 
\label{tab:rand_token_model} \\
\toprule
\textbf{Forum} &  \textbf{Train RMSE} &  \textbf{Test RMSE} &  \textbf{Test Gain (\%)} &  \textbf{Time (s)} & \textbf{Elastic Param} &  \textbf{Reg'tion Param} \\
\midrule
Buddhism      &              2.10 &           2.27 &           -0.00 &          221.30 &               1.0 &               1.0 \\
Economics     &              1.86 &           3.34 &           -0.00 &          201.97 &               1.0 &               1.0 \\
Fitness       &              2.18 &           2.10 &           -0.00 &          190.30 &               1.0 &               1.0 \\
Health        &              2.11 &           2.01 &           -0.00 &          186.69 &               1.0 &               1.0 \\
Interpersonal &             15.06 &          25.87 &           -3.65 &          358.41 &               1.0 &               1.0 \\
\bottomrule
\end{longtable}\begin{center} Source: Own calculations in PySpark\end{center}

\normalsize

The results of using unigram text of question titles and bodies is
displayed in table \ref{tab:rand_token_model}. This model struggles
particularly, only predicting means of the training set questions'
\texttt{Score} for every community except for Interpersonal, where it
actually performs worse than just predicting the mean by 3.5\%.
\textbf{What does this imply???}

Interestingly, the grid search gives an elastic parameter of 1 and
regularisation parameter of 1 for all models.

\footnotesize

\begin{longtable}[htbp] {@{} lcccccc @{}} 
\caption{\textbf{Global and Local Topic Model}} 
\label{tab:rand_topic_model} \\
\toprule
\textbf{Forum} &  \textbf{Train RMSE} &  \textbf{Test RMSE} &  \textbf{Test Gain (\%)} &  \textbf{Time (s)} & \textbf{Elastic Param} &  \textbf{Reg'tion Param} \\
\midrule
Buddhism      &              2.05 &           2.25 &            0.88 &           11.36 &              0.01 &               1.0 \\
Economics     &              1.86 &           3.34 &           -0.00 &            9.98 &              1.00 &               1.0 \\
Fitness       &              2.18 &           2.10 &           -0.00 &           12.33 &              1.00 &               1.0 \\
Health        &              2.11 &           2.01 &           -0.00 &           14.49 &              1.00 &               1.0 \\
Interpersonal &             22.31 &          24.95 &            0.04 &           11.47 &              1.00 &               1.0 \\
\bottomrule
\end{longtable}\begin{center} Source: Own calculations in PySpark\end{center}

\normalsize

It looks like predicting the score variable using just the textual
content of questions is not going well. What we have garnered is that
the fora are very heterogenous, having seen large differences in both
the descriptive statistics and predictive results - different parameters
come out as optimal for the length and topic models.

\footnotesize

\begin{longtable}[htbp] {@{} lcccccc @{}} 
\caption{\textbf{Length and Topic Model}} 
\label{tab:rand_final_model} \\
\toprule
\textbf{Forum} &  \textbf{Train RMSE} &  \textbf{Test RMSE} &  \textbf{Test Gain (\%)} &  \textbf{Time (s)} & \textbf{Elastic Param} &  \textbf{Reg'tion Param} \\
\midrule
Buddhism      &             2.04 &          2.24 &           1.32 &          11.24 &             0.01 &              1.0 \\
Economics     &             1.86 &          3.34 &          -0.00 &           8.96 &             1.00 &              1.0 \\
Fitness       &             2.18 &          2.10 &          -0.00 &           8.26 &             1.00 &              1.0 \\
Health        &             2.08 &          2.00 &           0.50 &           8.27 &             0.01 &              1.0 \\
Interpersonal &            22.26 &         24.87 &           0.36 &           8.44 &             1.00 &              1.0 \\
\bottomrule
\end{longtable}\begin{center} Source: Own calculations in PySpark\end{center}

\normalsize

\subsection{Temporal Train/Test Split}\label{temporal-traintest-split}

\footnotesize

\begin{longtable}[htbp] {@{} lcccccc @{}} 
\caption{\textbf{Length and Topic Model For Temporal Train/Test Split}} 
\label{tab:time_token_model} \\
\toprule
\textbf{Forum} &  \textbf{Train RMSE} &  \textbf{Test RMSE} &  \textbf{Test Gain (\%)} &  \textbf{Time (s)} & \textbf{Elastic Param} &  \textbf{Reg'tion Param} \\
\midrule
Buddhism      &             2.35 &          2.00 &           0.99 &          17.09 &             0.01 &             1.00 \\
Economics     &             3.08 &          2.25 &          -1.35 &          11.54 &             0.01 &             1.00 \\
Fitness       &             2.11 &          2.13 &          -0.47 &           9.23 &             1.00 &             0.01 \\
Health        &             1.93 &          2.16 &          -0.00 &           8.43 &             0.01 &             1.00 \\
Interpersonal &            25.57 &         22.02 &          -0.55 &           8.27 &             1.00 &             1.00 \\
\bottomrule
\end{longtable}\begin{center} Source: Own calculations in PySpark\end{center}

\normalsize

While I do not employ time-series models, I leave it to further research
to incorporate a way to also ``remember'' which questions are good, so
that in future there are no duplicates.

\textbf{EVEN AFTER GETTING RID OF A SUBSTANTIAL AMOUNT OF DATA FOR
CERTAIN DATASETS BY USING ONLY A YEAR WORTH OF DATA in the early stages
of the communitites' existence, THE TEMPORAL MODEL STILL STRUGGLES
SUBSTANTIALLY.}

\newpage

\section{\texorpdfstring{Recommendations for Further Research
\label{Recom}}{Recommendations for Further Research }}\label{recommendations-for-further-research}

A number of areas for further research stand out from the methodology I
developed here. Firstly, there are still many more complex features that
can be derived from question content alone that were not included in the
models in this research. Word-embeddings (({\textbf{???}})) might be
\textbf{more} successful in predicting community engagement metrics.

As discussed \textbf{in detail} in section \ref{Vars}, there are also a
myriad of other options for community engagement besides the
\texttt{Score} variable, each with their own advantages and
disadvantages. While I believe I thoroughly justified and validated my
choice of the continuous \texttt{Score} variable as a comprehensive and
objective response variable, not least because accurate predictions of
it would be \textbf{highly useful information for questioners wishing to
improve their questions}, a thorough exploration and predictive modeling
of other measurements of community engagement with the models developing
here would be extremely valuable.

Another area previously discussed that is ripe for further research is
the editing of questions. As a reminder, questions can be edited not
only by the original poster, but also by anyone with \textbf{2000}
reputation or more. One suggestion for further research would be
investigating how much editing takes place over questions, in what
average timeframe edits are completed compared to votes cast and views
accumulated, as well as how evenly editing is distributed over
questions. This research would then be able to test my assumption that
most edits, if any, take place before the majority of votes and views
are recorded.

There are also finer nuances regarding the functioning of the
StackExchange sites, \textbf{some of which were} discussed in section
\ref{Issues}. For registered users in various communities, there remains
some confusion on when to up-vote and down-vote questions
(\url{https://meta.stackexchange.com/questions/12772/should-i-upvote-bad-questions}).
This links with how there are potentially vastly different motivations
behind voting? Over time various communities have also implemented
different interventions to nudge users to better formulate and structure
their questions, i.e.~reminders of doing prior research, including
reproducible code for programming websites, and even going as far as to
check that the \texttt{Title} of new questions do not match previous
questions too closely for fear of allowing a duplicate question to be
asked in the community. These nudges would no doubt affect the
distribution and evolution of questions temporally in the data, and
consequently affect metrics such as \texttt{Score} and
\texttt{ViewCount}, which links with the next final recommendation for
further research.

\textbf{Most} importantly as hinted throughout and demonstrated at the
end of this analysis, temporality of the data is something that needs to
be taken into account. This is at least true in the sense that questions
which have existed longer in communities trivially would have had more
time to accumulate votes and views, but my further hypothesis is that
the variation in community engagement metrics has decreased
substantially over time as users and communities have refined how they
permit and value certain questions - this new hypothesis of
heterogeneity over time can be tested with variance equivalence testing
for samples across time. All of this suggests that any model aiming to
predict future community engagement in online Q\&A fora must be expanded
to include temporal effects and time-series elements.

Even if the temporality issue was solved however, another challenge is
getting the model to pick up duplicate questions (which are
ill-considered in all communities). This would mean instilling in the
model that a question can be similar enough to previous questions for
the model to learn that it's a good question, however not too similar so
that the community perceives it as a duplicate, assumes a ``lack of
prior research'' on the questioner's part and then reacts negatively to
the question. As one can see therefore, there is still much work to be
done in order to accurately predict future communtity engagement in
online Q\&A fora.

\newpage

\section{\texorpdfstring{Concluding Remarks
\label{Concl}}{Concluding Remarks }}\label{concluding-remarks}

The aim of this research was to predict the range of positive/negative
community engagement that questions elicit, with the practical
application of providing this information to questioners so that they
can improve their questions before adding demand to a community. I
believe that no prior research has endeavoured with the methodology here
in this respective framework to predict and capture community
engagement. At the very least, the research here has improved upon the
extent of how community engagement can be ascertained from online Q\&A
communities, and has yielded insight into how homogeneously community
engagement exists over diverse communities with various subject matter.
I believe that using this tool, online Q\&A users will be assisted in
improving their submitted questions which will enhance the productivity
of all online Q\&A communities wholly. Furthermore, room exists for
implementation on any assortment of Q\&A sites, counting Massive Open
Online Courses.

\textbf{There is much heterogeneity in the data I have analysed, not
only across fora but also over time. This, as well as the fact that I
have attempted to predict on a continuous variable rather than binary,
makes the problem of predicting community engagement from text-only data
substantially more difficult, as can be seen from the poor predictive
performance of the models employed. I leave it to future research to
employ more sophisticated time-series models to capture temporal
effects/features from the data.}

\newpage

\section*{References}

\hypertarget{refs}{}
\hypertarget{ref-Agichtein2008}{}
Agichtein, E. \emph{et al.} (2008) `Finding high-quality content in
social media', in \emph{Proceedings of the 2008 international conference
on web search and data mining}. ACM, pp. 183--194. doi:
\href{https://doi.org/10.1145/1341531.1341557}{10.1145/1341531.1341557}.

\hypertarget{ref-Allamanis2013}{}
Allamanis, M. and Sutton, C. (2013) `Why, when, and what: Analyzing
stack overflow questions by topic, type, and code', in \emph{2013 10th
working conference on mining software repositories (msr)}. IEEE, pp.
53--56. doi:
\href{https://doi.org/10.1109/MSR.2013.6624004}{10.1109/MSR.2013.6624004}.

\hypertarget{ref-Anderson2012}{}
Anderson, A. \emph{et al.} (2012) `Discovering value from community
activity on focused question answering sites: a case study of stack
overflow', in \emph{Proceedings of the 18th acm sigkdd international
conference on knowledge discovery and data mining}. ACM, pp. 850--858.
Available at: \url{http://dl.acm.org/citation.cfm?id=2339665}.

\hypertarget{ref-Bian2009}{}
Bian, J. \emph{et al.} (2009) `Learning to recognize reliable users and
content in social media with coupled mutual reinforcement', in
\emph{Proceedings of the 18th international conference on world wide
web}. ACM, pp. 51--60. doi:
\href{https://doi.org/10.1145/1526709.1526717}{10.1145/1526709.1526717}.

\hypertarget{ref-Blei2003}{}
Blei, D. M., Ng, A. Y. and Jordan, M. I. (2003) `Latent Dirichlet
Allocation', \emph{Journal of Machine Learning Research}, 3, pp.
993--1022.

\hypertarget{ref-Chiang2010}{}
Chiang, D. \emph{et al.} (2010) `Bayesian Inference for Finite-State
Transducers', in \emph{Human language technologies: The 2010 annual
conference of the north american chapter of the association for
computational linguistics}. Association for Computational Linguistics
(June), pp. 447--455. Available at:
\href{http://www.isi.edu/\%7B~\%7Dsravi/pubs/naacl2010\%7B/_\%7Dbayes-fst.pdf}{http://www.isi.edu/\{\textasciitilde{}\}sravi/pubs/naacl2010\{\textbackslash{}\_\}bayes-fst.pdf}.

\hypertarget{ref-Fligner1986}{}
Fligner, M. and Verducci, J. S. (1986) `Distance based ranking models',
\emph{Journal of the Royal Statistical Society: Series B
(Methodological)}, 48(3), pp. 359--369.

\hypertarget{ref-Jeon2006}{}
Jeon, J. \emph{et al.} (2006) `A framework to predict the quality of
answers with non-textual features', in \emph{Proceedings of the 29th
annual international acm sigir conference on research and development in
information retrieval}. ACM, pp. 228--235. doi:
\href{https://doi.org/10.1145/1148170.1148212}{10.1145/1148170.1148212}.

\hypertarget{ref-Li2010}{}
Li, B. and King, I. (2010) `Routing questions to appropriate answerers
in community question answering services', in \emph{Proceedings of the
19th acm international conference on information and knowledge
management}. ACM, pp. 1585--1588. doi:
\href{https://doi.org/10.1145/1871437.1871678}{10.1145/1871437.1871678}.

\hypertarget{ref-Li2012}{}
Li, B. \emph{et al.} (2012) `Analyzing and predicting question quality
in community question answering services', in \emph{Proceedings of the
21st international conference on world wide web}. ACM, pp. 775--782.
doi:
\href{https://doi.org/10.1145/2187980.2188200}{10.1145/2187980.2188200}.

\hypertarget{ref-Li2011}{}
Li, B., King, I. and Lyu, M. R. (2011) `Question routing in community
question answering', in \emph{Proceedings of the 20th acm international
conference on information and knowledge management}. ACM, pp.
2041--2044. doi:
\href{https://doi.org/10.1145/2063576.2063885}{10.1145/2063576.2063885}.

\hypertarget{ref-Liu2008}{}
Liu, Y., Bian, J. and Agichtein, E. (2008) `Predicting information
seeker satisfaction in community question answering', in
\emph{Proceedings of the 31st annual international acm sigir conference
on research and development in information retrieval}. ACM (Section 2),
pp. 483--490. doi:
\href{https://doi.org/10.1145/1390334.1390417}{10.1145/1390334.1390417}.

\hypertarget{ref-Qu2009}{}
Qu, M. \emph{et al.} (2009) `Probabilistic question recommendation for
question answering communities', in \emph{Proceedings of the 18th
international conference on world wide web}. ACM (2), pp. 1229--1230.
doi:
\href{https://doi.org/10.1145/1526709.1526942}{10.1145/1526709.1526942}.

\hypertarget{ref-Ravi2014}{}
Ravi, S. \emph{et al.} (2014) `Great Question! Question Quality in
Community Q\&A.', in \emph{Eighth international aaai conference on
weblogs and social media}. (1), pp. 426--435.

\hypertarget{ref-Riahi2012}{}
Riahi, F. \emph{et al.} (2012) `Finding expert users in community
question answering', in \emph{Proceedings of the 21st international
conference on world wide web}. ACM, pp. 791--798. doi:
\href{https://doi.org/10.1145/2187980.2188202}{10.1145/2187980.2188202}.

\hypertarget{ref-Shah2010}{}
Shah, C. and Pomerantz, J. (2010) `Evaluating and predicting answer
quality in community QA', in \emph{Proceedings of the 33rd international
acm sigir conference on research and development in information
retrieval}. ACM (March 2008), pp. 411--418. doi:
\href{https://doi.org/10.1145/1835449.1835518}{10.1145/1835449.1835518}.

\hypertarget{ref-Shah2018}{}
Shah, V. \emph{et al.} (2018) `Adaptive matching for expert systems with
uncertain task types', in \emph{2017 55th annual allerton conference on
communication, control, and computing (allerton)}. IEEE, pp. 753--760.
doi:
\href{https://doi.org/10.1109/ALLERTON.2017.8262814}{10.1109/ALLERTON.2017.8262814}.

\hypertarget{ref-Sung2013}{}
Sung, J., Lee, J.-g. and Lee, U. (2013) `Booming Up the Long Tails:
Discovering Potentially Contributive Users in Community-Based Question
Answering Services', in \emph{Seventh international aaai conference on
weblogs and social media}, pp. 602--610.

\hypertarget{ref-Szpektor2013}{}
Szpektor, I., Maarek, Y. and Pelleg, D. (2013) `When relevance is not
enough: promoting diversity and freshness in personalized question
recommendation', in \emph{Proceedings of the 22nd international
conference on world wide web}. ACM, pp. 1249--1260.

\hypertarget{ref-Tian2013}{}
Tian, Q., Zhang, P. and Li, B. (2013) `Towards Predicting the Best
Answers in Community-Based Question-Answering Services', in
\emph{Seventh international aaai conference on weblogs and social
media}, pp. 725--728.

\hypertarget{ref-Wu2008}{}
Wu, H., Wang, Y. and Cheng, X. (2008) `Incremental probabilistic latent
semantic analysis for automatic question recommendation', in
\emph{Proceedings of the 2008 acm conference on recommender systems}.
ACM, p. 99. doi:
\href{https://doi.org/10.1145/1454008.1454026}{10.1145/1454008.1454026}.

\hypertarget{ref-Zhou2012}{}
Zhou, T. C., Lyu, M. R. and King, I. (2012) `A classification-based
approach to question routing in community question answering', in
\emph{Proceedings of the 21st international conference on world wide
web}. ACM, pp. 783--790. Available at:
\url{http://www2012.wwwconference.org/proceedings/companion/p783.pdf}.

% code for wordcount (INCOMPLETE)
\newcommand\wordcount{
    \immediate\write18{texcount -sub=section \jobname.tex  | grep "Section" |     sed -e 's/+.*//' | sed -n \thesection p > 'count.txt'}
(\input{count.txt}words)}

\end{document}
